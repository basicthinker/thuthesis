\begin{resume}

  \resumeitem{个人简历}

  1986 年 9 月 12 日出生于河北省保定市。

  2006 年 9 月考入东北师范大学软件学院软件工程专业,2010 年 7 月本科毕业并获得工学学士学位。

  2010 年 9 月免试进入清华大学计算机科学与技术系攻读硕士学位;2013 年 7 月转为提前攻读博士学位至今。

  博士在读期间,2013 年 11 月至 2014 年 5 月在卡内基梅隆大学(Carnegie Mellon University)做访问学者;2014 年 10 月至 2015 年 8 月在斯坦福大学(Stanford University)做访问研究员。

  \researchitem{发表的学术论文} % 发表的和录用的合在一起
% 学位论文写作指南:
% 在学期间发表的学术论文分以下三部分按顺序分别列出,每部分之间空 1
% 行,序号可连续排列
% 1. 已经刊载的学术论文(本人是第一作者,或者导师为第一作者本人是第二作者)
% 2. 尚未刊载,但已经接到正式录用函的学术论文(本人为第一作者,或者
% 导师为第一作者本人是第二作者)。
% 3. 其他学术论文。可列出除上述两种情况以外的其他学术论文,但必须是
% 已经刊载或者收到正式录用函的论文。
  \begin{publications}
  \item \textbf{Jinglei Ren}, Jishen Zhao, Samira Khan, Jongmoo Choi, Yongwei Wu, and Onur Mutlu.
Cooperative Checkpointing: A Software-Transparent Mechanism for Supporting Crash Consistency in Persistent Memory Systems, 
Proceedings of the 48th Annual IEEE/ACM International Symposium on Microarchitecture (MICRO), Dec. 2015.(CCF 推荐 A 类会议)
  \item \textbf{Jinglei Ren}, Chieh-Jan Mike Liang, Yongwei Wu, and Thomas Moscibroda.
Memory-Centric Data Storage for Mobile Systems, 
Proceedings of the 2015 USENIX Annual Technical Conference (USENIX ATC), Jul. 2015.(CCF 推荐 B 类会议)
  \item Mingxing Zhang, Yongwei Wu, Shan Lu, Shanxiang Qi, \textbf{Jinglei Ren}, and Weimin Zheng.
AI: A Lightweight System for Tolerating Concurrency Bugs 
Proceedings of the 22nd ACM SIGSOFT International Symposium on the Foundations of Software Engineering (FSE), Nov. 2014.(CCF 推荐 A 类会议)
  \item \textbf{Jinglei Ren}, Yongwei Wu, Meiqi Zhu, and Weimin Zheng.
Quatrain: Accelerating Data Aggregation Between Multiple Layers,
IEEE Transactions on Computers (TC), Vol. 63, No. 5, pp. 1207 - 1219, May 2014. (CCF 推荐 A 类期刊,SCI 检索)
  \item Yongwei Wu, Weichao Guo, \textbf{Jinglei Ren}, Xun Zhao, and Weiming Zheng.
NO2: Speeding Up Parallel Processing of Massive Compute-Intensive Tasks,
IEEE Transactions on Computers (TC), Vol. 63, No. 10, pp. 2487 - 2499, Oct. 2014.(CCF 推荐 A 类期刊,SCI 检索)
  \end{publications}

  \researchitem{研究成果} % 有就写,没有就删除
  \begin{achievements}
  \item 武永卫,任晶磊. : 中国, CNxxx. (中国专利公开号.)
  \item J. Ren, T. Moscibroda, M. Liang. : USA, No.xx/xxx, xxx. (美国发明专利申请号.)
  \end{achievements}
\end{resume}
