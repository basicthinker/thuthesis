%%% Local Variables:
%%% mode: latex
%%% TeX-master: "../main"
%%% End:

% 如果使用声明扫描页,将可选参数指定为扫描后的 PDF 文件名,例如:
%\begin{ack}[scan-statement.pdf]
\begin{ack}

衷心感谢导师郑纬民教授和武永卫教授。他们在学术上悉心指导,在生活上关心照顾,为学生的发展构建了广阔的平台。忘不了和老师们的一次次讨论,忘不了老师们的谆谆教诲,也忘不了武老师在我的婚礼上作为主婚人的温馨致辞。

我还要感谢以往论文的其他合作者和指导者,他们的智慧亦融合在我的毕业论文中。微软亚洲研究院的Thomas Moscibroda和Mike Liang研究员为MobiFS项目的策略设计给予了宝贵建议。卡內基·梅隆大学(Carnegie Mellon University)的Onur Mutlu教授帮助我挖掘想法中价值,和时任惠普实验室(HP Labs)研究员的Jishen Zhao博士、时任英特尔研实验室(Intel Labs)研究员的Samira Khan博士一起反复修改我的论文。斯坦福大学(Stanford University)的David Cheriton教授和Heiner Litz博士多次斧正我的想法,他们组的工作为我的论文提供了很好的基础。清华实验室的师兄弟刘立坤、张扬、章明星、王博、郭维超等同学,不断和我分享着他们的创见、努力和一起玩乐的美好时光。此外,我合作的论文中,当时威斯康星大学麦迪逊分校(University of Wisconsin–Madison)的Shan Lu教授给予的帮助和指导,在此一并感谢;原SanDisk副总裁John Butsh博士对论文工作的判断和鼓励也令我受益匪浅。

感谢我的妻子闫睿颖女士,她无私无畏的支持和先于我工作所给予的经济援助,帮我度过了无数难关。儿子任念初的到来,赐予我生命的灵感和前行的动力。更要感谢我的父母任东兴、任焕茹,我的岳父母闫子政、张世平,他们构筑了我多年科研工作的坚实后方,是我毕业论文能够成文的不可磨灭的功臣。

最后,感谢自然科学基金、863计划、龙门教育基金、登峰基金等对本人工作的资助。感谢答辩委员会专家及匿名评审专家的宝贵时间和指导。

\end{ack}
