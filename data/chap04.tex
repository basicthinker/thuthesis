\chapter{双模式检查点生成技术及混合内存性能优化}
\label{chap:thnvm}

\section{软件透明的一致性保证机制}

新兴的字节编址的(byte-addressable)非易失性内存介质,诸如STT-RAM~\cite{4443191}、PCM~\cite{Raoux:2008:PRA}和ReRAM~\cite{5607274}等预示着持久化内存系统,一个在内存和存储栈上的新的中间层。持久化内存系统融合了传统内存系统(快速的load/store指令接口)和存储系统(数据持久化)的优良特性,模糊了两者之间的界限。它给应用带来的一个重要的益处是,可以通过load/store指令高效地直接地访问内存中的持久化数据,而不需要与存储设备进出换页、在(反)序列化时更改数据格式以及触发臃肿的系统调用。

持久化内存相对于传统的易失性内存系统引入了一个关键的要求,系统故障时的一致性保证~\cite{Lamb:1991:ODS,Copeland:1989:CSR,Shapiro:1999:EFC}。这要求持久化内存系统,在掉电或系统失效等故障出现时,依然能够保证数据的一致性不受部分或是乱序的NVM写的影响。我们以原子性地更新保存在NVM上的数据结构A和B为例,总会有对A或B的一个更改会先于其他写入NVM。如果系统在一个更改完成之后出现故障或掉电,两个数据结构可能处于不一致的中间状态,只包含部分更改。对于传统易失性内存,这不是一个问题,因为程序恢复后内存中的数据都会丢失。但对于持久化内存,这些不一致的数据会一直保留。所以,持久化内存系统需要保证保存到NVM的数据可以在系统重启后恢复到一个一致的状态,即维持数据的一致性。维持数据的一致性原本仅是对磁盘或闪存等存储系统的要求,但将数据持久化引入内存后,它即成为内存系统同样面临的一个挑战。

大多数之前的持久化内存设计依赖于程序员的人工努力来保证系统故障时的数据一致性。应用的开发人员需要显式地使用特定的编程模型和软件接口来访问和操作内存中的持久化数据。这些软件接口构建于为持久化内存定制的运行时库~\cite{Condit:2009:BIT:1629575.1629589,Volos:2011:MLP:1950365.1950379}或硬件架构~\cite{Zhao:2013:KCP:2540708.2540744}。这种设计似乎给程序员提供了对数据持久化的完全控制,但要求程序员管理持久化内存会带来至少三方面不良影响。首先,应用开发者必须用新的编程接口API来实现程序或者深度更改遗留代码,通常要显式地声明和划分持久的和临时数据结构。其次,大多数之前的持久化内存设计需要使用事务性内存控制版本和对NVM写的顺序,而事务性内存的扩展性依然面临挑战。第三,现在持久性内存的编程模型可能基于很多种语义和软件接口实现,会带来兼容性和移植性问题。为此,我们工作的目标是设计高效的故障时数据一致性保障机制,使得持久化内存的使用范围得到扩展而不必需要复杂的应用更改。

我们提出了THNVM,一种新的支持对软件透明的故障时数据一致性的持久化内存设计。它允许基于事务的和基于锁的程序直接在持久化内存硬件上运行。THNVM使用DRAM和NVM混合的架构和高效的硬件辅助的检查点生成技术来达到DRAM水平的性能。这里的一个关键挑战是减少由于生成检查点而导致的应用停滞时间。我们发现,在停滞时间和元数据存储开销之间存在一个折衷。在较小的粒度上生成检查点,带来的停滞时间可以比较短,而对应元数据占用的空间却会非常大;在较大的粒度上生成检查点,可以降低元数据占用的空间,但是会导致较长的停滞时间。因此,单一的检查点生成技术(或者基于小粒度或者基于大粒度)难以达到最优的效果。为了解决这一问题,我们提出了双模式检查点生成机制,它可以同步地为分散的和集中的内存写分别基于CPU缓存块粒度和操作系统页粒度生成检查点。与先前使用写时拷贝(copy-on-write,COW)或日志技术的持久化内存设计相比,THNVM显著减小了存储的损耗并增加了内存带宽的利用率。

特别地,本章工作做出了如下几点贡献:(1)我们提出了一种新的持久化内存设计,支持对软件透明的故障时数据一致性保障。我们的设计允许基于事务的和基于锁的应用通过load/store指令接口使用持久化内存。(2)对于生成检查点的粒度,我们定位了程序停滞时间和元数据空间占用之间的权衡关系。在任意一个粒度上追踪对持久化内存的更改都不是最优的。(3)我们设计了一个新的双模式检查点生成技术。我们的技术比页粒度的检查点生成机制减少了86.2\%的停滞时间,而仅比缓存块粒度的检查点生成机制多用26\%的内存控制器硬件空间。(4)我们在内存模拟器上实现了双模式检查点生成机制,使用我们经过形式化证明的一致性协议。它可以在保证数据一致性的同时,将检查点生成过程和程序执行重叠。我们的解决方案可以达到不提供一致性支持的全DRAM系统性能的95.1\%。

\section{现有技术缺陷}

THNVM的设计有三个本质特性:软件透明,检查点生成,以及双模式。本节讨论我们开发这三种特性的原因。

\subsection{软件一致性保障技术的缺陷}

\begin{figure}[t]
\centering \includegraphics[width=\columnwidth]{motivation-code}
  \caption{实例代码:向持久化的哈希表中插入一个项。分别采用(a)传统的软件接口或(b)软件透明的THNVM。}
\label{fig:motivation-code}
\end{figure}

之前大多数持久化内存设计~\cite{Condit:2009:BIT:1629575.1629589,Volos:2011:MLP:1950365.1950379,Coburn:2011:NMP:1950365.1950380,Zhao:2013:KCP:2540708.2540744,Venkataraman:2011:CDD:1960475.1960480}要求应用开发者显式地定义持久性数据并使用特定库来访问数据——故障时数据一致性保障机制与软件的语义紧耦合。我们通过图\ref{fig:motivation-code}中的实例代码来展示这种持久化内存设计的不足。该图比较了一个更新持久性哈希表项的软件方法的两种实现(代码改编自STAMP~\cite{Cao:2008:STA})。图\ref{fig:motivation-code}(a)展示了使用类似于现有设计~\cite{Condit:2009:BIT:1629575.1629589, Volos:2011:MLP:1950365.1950379}的软件事务的实现;而图\ref{fig:motivation-code}(b)中的实例代码采用了THNVM的软件透明的故障时数据一致性支持。我们可以看到,在支持数据一致性时涉及软件支持,会有多方面的不足。

首先,划分临时性和持久性数据很麻烦而且容易出错。图\ref{fig:motivation-code}(a)展示了许多无法避免的标记。程序员需要对软件行为有深入的了解,并且小心地确定哪些数据结构需要持久化以及如何操控这些数据。例如,图\ref{fig:motivation-code}(a)的第3行仅当这个哈希表实现固定数目的桶时才是正确的。

其次,在一个统一的地址空间内,临时性和持久性数据之间的引用需要小心的管理。然而,对该问题的基于软件的解决方案并非最佳。例如,为了处理NV-to-V指针(非易失性指针指向易失性数据,图\ref{fig:motivation-code}(a)的第8行),NV-heaps~\cite{Coburn:2011:NMP:1950365.1950380}会简单地抛出一个运行时错误以禁用此类指针。

第三,应用需要使用事务,通常是通过事务性内存接口(例如被Mnemosyne~\cite{Volos:2011:MLP:1950365.1950379}使用的TinySTM~\cite{Felber:2008:DPT}),来更新持久化数据,如图\ref{fig:motivation-code}(a)所示。遗憾的是,事务性内存,不论基于硬件还是基于软件,存在多方面的可扩展性问题~\cite{Cascaval:2008:STM:1454456.1454466,Pankratius:2011:STM:1989493.1989500,Dice:2009:EEC:1508244.1508263}。此外,开发人员必须使用新的API来实现或者重写各种代码库和程序,需要不可忽略的人工投入。不熟悉这些新API的程序员需要一个痛苦的学习过程。

最后,应用需要构建在特定的库之上,例如libmnemosyne~\cite{Volos:2011:MLP:1950365.1950379}和
NV-heaps~\cite{Coburn:2011:NMP:1950365.1950380}。这会很大程度上降低持久化内存应用的移植性——在一个库上实现的应用,如果想采用其他的库,需要重新实现。编译器确实可以在一定程度上减轻程序员实现和移植代码的负担。然而,编译器会无差别地插装所有内存的读和写操作,带来可见的性能损耗。我们在GCC libitm~\cite{libitm}上实验了STAMP事务性基准测试~\cite{Cao:2008:STA},结果插装本身最多可带来高达8\%的性能损失。

我们采用了允许软件程序直接访问持久化内存的使用模型。如图\ref{fig:motivation-code}(b)所示,我们可以不更改代码而实现数据结构的持久化。

\subsection{日志及写时拷贝技术的缺陷}



\subsection{粒度的权衡}

\section{双模式检查点生成技术}

\subsection{系统定义}

\subsection{基于块重映射的检查点生成}

\subsection{基于页回写的检查点生成}

\subsection{双重检查点生成机制}

\section{一致性机制的状态机表达}

\section{一致性机制的正确性证明}

\section{系统实现}

\subsection{地址空间管理}

\subsection{元数据管理}

\subsection{数据保存}

\subsection{数据恢复}

\section{系统测评}

\subsection{实验设置}

\subsection{微基准测试}

\subsection{存储类基准测试}

\subsection{计算类基准测试}

\subsection{敏感度分析}

\section{本章小结}

