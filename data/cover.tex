%%% Local Variables:
%%% mode: latex
%%% TeX-master: t
%%% End:
\secretlevel{公开}

\ctitle{非易失性内存系统的\\数据一致性机制和性能优化}
% 根据自己的情况选,不用这样复杂
\makeatletter
\ifthu@bachelor\relax\else
  \ifthu@doctor
    \cdegree{工学博士}
  \else
    \ifthu@master
      \cdegree{工学硕士}
    \fi
  \fi
\fi
\makeatother


\cdepartment[计算机]{计算机科学与技术系}
\cmajor{计算机科学与技术}
\cauthor{任晶磊}
\csupervisor{郑纬民教授}
% 如果没有副指导老师或者联合指导老师,把下面两行相应的删除即可。
%\cassosupervisor{陈文光教授}
%\ccosupervisor{武永卫教授}
% 日期自动生成,如果你要自己写就直接改这个cdate。
% 硕博也可以启用如下三行,替换其中的\the\year和\the\month为阿拉伯数字。
\CTEXdigits{\zhyear}{2015}
\CTEXnumber{\zhmonth}{12}
\cdate{\zhyear{}年\zhmonth{}月}

% 博士后部分
% \cfirstdiscipline{计算机科学与技术}
% \cseconddiscipline{系统结构}
% \postdoctordate{2009年7月——2011年7月}

\etitle{Consistent and Efficient\\Persistence Mechanisms of\\Non-volatile Memory Systems}
% 这块比较复杂,需要分情况讨论:
% 1. 学术型硕士
%    \edegree:必须为Master of Arts或Master of Science(注意大小写)
%              “哲学、文学、历史学、法学、教育学、艺术学门类,公共管理学科
%               填写Master of Arts,其它填写Master of Science”
%    \emajor:“获得一级学科授权的学科填写一级学科名称,其它填写二级学科名称”
% 2. 专业型硕士
%    \edegree:“填写专业学位英文名称全称”
%    \emajor:“工程硕士填写工程领域,其它专业学位不填写此项”
% 3. 学术型博士
%    \edegree:Doctor of Philosophy(注意大小写)
%    \emajor:“获得一级学科授权的学科填写一级学科名称,其它填写二级学科名称”
% 4. 专业型博士
%    \edegree:“填写专业学位英文名称全称”
%    \emajor:不填写此项
\edegree{Doctor of Philosophy}
\emajor{Computer Science and Technology}
\eauthor{Jinglei Ren}
\esupervisor{Professor Weimin Zheng}
%\eassosupervisor{Professor Yongwei Wu}
\edate{December, 2015}

% 定义中英文摘要和关键字
\begin{cabstract}
    新型非易失性内存介质,诸如闪存(flash)、相变内存(phase-change
memory,PCM)、可变电阻式内存(ReRAM)等,可提供传统硬盘等外部存储器的数据持久化能力,和日益接近动态随机访问存储器(DRAM)等内部存储器的存取性能。非易失性内存介质及其软硬件系统,可以融合传统易失性内部存储和持久化外部存储的优良特性,提升上层应用软件和系统整体的效能。

    非易失性内存系统使得内存数据在系统故障断电后依然可以得到特定程度的保留。该特性在减少传统持久化机制带来的性能损耗方面作用显著,但于此同时,也使得数据一致性问题尤为突出。为保证数据一致性,往往需要对上层应用程序访问内存的接口加以限制。数据一致性机制及其应用程序接口方式,在非易失性内存系统性能、易用性及两者间的权衡等方面扮演者至关重要的角色。

    本论文系统研究了在非易失性内存系统使用文件系统、事务性内存和软件透明三种主要访存接口方式的情况下,数据一致性的保护机制,以及针对特定负载的系统性能优化策略。研究成果的主要创新点包括:

  \begin{itemize}
    \item 将原子性事务(atomic transaction)机制引入操作系统页缓存,使得内存数据持久化时能够保证一致性;将该项技术用于手机文件系统,依据不同应用和用户的I/O特性,提出了优化手机能耗和应用响应的新度量和新模型。
    \item 。
    \item 。
  \end{itemize}

\end{cabstract}

\ckeywords{内存, 非易失性, 持久化, 事务内存, 检查点}

\begin{eabstract}
   An abstract of a dissertation is a summary and extraction of research work
   and contributions. Included in an abstract should be description of research
   topic and research objective, brief introduction to methodology and research
   process, and summarization of conclusion and contributions of the
   research. An abstract should be characterized by independence and clarity and
   carry identical information with the dissertation. It should be such that the
   general idea and major contributions of the dissertation are conveyed without
   reading the dissertation.

   An abstract should be concise and to the point. It is a misunderstanding to
   make an abstract an outline of the dissertation and words ``the first
   chapter'', ``the second chapter'' and the like should be avoided in the
   abstract.

   Key words are terms used in a dissertation for indexing, reflecting core
   information of the dissertation. An abstract may contain a maximum of 5 key
   words, with semi-colons used in between to separate one another.
\end{eabstract}

\ekeywords{memory, persistence, non-volatility, transactional memory,
checkpointing}

