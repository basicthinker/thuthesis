%%% Local Variables:
%%% mode: latex
%%% TeX-master: t
%%% End:
\secretlevel{公开}

\ctitle{持久性内存系统中\\高效的数据一致性机制研究}
% 根据自己的情况选,不用这样复杂
\makeatletter
\ifthu@bachelor\relax\else
  \ifthu@doctor
    \cdegree{工学博士}
  \else
    \ifthu@master
      \cdegree{工学硕士}
    \fi
  \fi
\fi
\makeatother


\cdepartment[计算机]{计算机科学与技术系}
\cmajor{计算机科学与技术}
\cauthor{任晶磊}
\csupervisor{郑纬民教授}
% 如果没有副指导老师或者联合指导老师,把下面两行相应的删除即可。
%\cassosupervisor{教授}
%\ccosupervisor{教授}
% 日期自动生成,如果你要自己写就直接改这个cdate。
% 硕博也可以启用如下三行,替换其中的\the\year和\the\month为阿拉伯数字。
\CTEXdigits{\zhyear}{2015}
\CTEXnumber{\zhmonth}{12}
\cdate{\zhyear{}年\zhmonth{}月}

% 博士后部分
% \cfirstdiscipline{计算机科学与技术}
% \cseconddiscipline{系统结构}
% \postdoctordate{2009年7月——2011年7月}

\etitle{Efficient Mechanisms for\\Supporting Crash Consistency in\\Persistent Memory Systems}
% 这块比较复杂,需要分情况讨论:
% 1. 学术型硕士
%    \edegree:必须为Master of Arts或Master of Science(注意大小写)
%              “哲学、文学、历史学、法学、教育学、艺术学门类,公共管理学科
%               填写Master of Arts,其它填写Master of Science”
%    \emajor:“获得一级学科授权的学科填写一级学科名称,其它填写二级学科名称”
% 2. 专业型硕士
%    \edegree:“填写专业学位英文名称全称”
%    \emajor:“工程硕士填写工程领域,其它专业学位不填写此项”
% 3. 学术型博士
%    \edegree:Doctor of Philosophy(注意大小写)
%    \emajor:“获得一级学科授权的学科填写一级学科名称,其它填写二级学科名称”
% 4. 专业型博士
%    \edegree:“填写专业学位英文名称全称”
%    \emajor:不填写此项
\edegree{Doctor of Philosophy}
\emajor{Computer Science and Technology}
\eauthor{Jinglei Ren}
\esupervisor{Professor Weimin Zheng}
%\eassosupervisor{Professor Yongwei Wu}
\edate{December, 2015}

% 定义中英文摘要和关键字
\begin{cabstract}

新型非易失性内存介质,诸如闪存(flash)、相变内存(phase-change memory,PCM)、可变电阻式内存(ReRAM)等,可提供传统硬盘等外部存储器的数据持久化能力,和日益接近动态随机访问内存(DRAM)等内部存储器的存取性能。非易失性内存介质及其软硬件系统共同构成\emph{持久性内存}(persistent memory),可以融合传统易失性内部存储和非易失性外部存储的优良特性,提升上层应用软件和系统整体的性能和效能。

持久性内存系统使得内存数据在系统故障发生后依然得以保留。该特性在减少传统持久化机制带来的性能损耗方面作用显著,但于此同时,也使得\emph{故障时数据一致性}(crash consistency)问题尤为突出。为保证故障时数据一致性,往往需要对上层应用程序访问内存的接口加以限制。数据一致性机制及其应用程序接口方式,在持久性内存系统性能、易用性及两者间的权衡等方面扮演者至关重要的角色。

%近年提出的持久性内存系统设计通常选用基于事务(transaction)的接口,例如事务性内存等。此类解决方案虽然利用了事务接口在原子性、一致性、隔离性和持久性等方面的良好定义,但与此同时限制了持久性内存的应用范围,面临可扩展性方面的挑战,并给应用开发人员带来很大的编程负担。

%本论文研究了在持久性内存系统中提供\emph{应用透明}的访问接口的问题,以达到传统应用程序不需要更改即可利用新型持久性内存的目的。对于按块访问的非易失性内存介质(如闪存),我们基于移动系统的独特应用环境,将传统动态内存和非易失性内存整体视作一层持久性内存系统,改进现有文件系统设计,不需要手机应用实现的更改,即可以获得响应度和能耗的显著改进。对于按字节访问的非易失性内存介质(如PCM),我们完全以传统处理器内存访问指令作为接口,避免现有面向持久性内存的程序对易失性数据和非易失性数据进行划分的强制要求,内含新的双模式检查点生成技术(dual-scheme checkpointing),消除了软件编程的负担和性能瓶颈,并可应用于不作修改的遗留代码。该设计需要独特的故障时数据一致性保证机制,为此我们设计了新的数据一致性协议,并给出了其正确性的形式化证明。

本论文研究了持久性内存系统在文件系统、事务性内存和软件透明三种主要数据存取接口方式下,如何设计和实现高效的故障时数据一致性保证机制。主要创新点和研究成果包括:

\begin{itemize}
\item \emph{文件系统接口下的多版本缓存事务技术。}将原子性事务(atomic transaction)机制引入操作系统页缓存,解决由动态内存和非易失性内存构成的持久性内存系统的故障时数据一致性问题。将该技术应用于移动系统环境,在新的持久性内存系统假设下,改进现有手机文件系统设计,提出优化手机能耗和应用响应的新指标和三组新策略算法。
\item \emph{事务性内存接口下的小缓冲区组技术。}根据NVM Express接口和固态硬盘的新特性,为事务性内存设计了高效的一致性的持久化机制,构成持久性内存的一种新的实现方式。该机制包含小缓冲器组(small buffer array)的设计,在保证故障时数据一致性的同时,显著降低组提交(group commit)中提交者相互等待的时间,兼得理想的吞吐量和延迟。
\item \emph{软件透明接口下的双模式检查点生成技术。}提出支持软件透明的故障时数据一致性的混合持久性内存设计,通过双模式检查点生成技术高效地生成一致的可恢复的检查点。该技术同时在缓存块粒度和页粒度上产生检查点,可使软件执行与产生检查点的延迟重合,大幅减少停滞时间,同时实现可行的硬件空间占用。
\item \emph{软件透明的数据一致性协议及其形式化证明。}双模式检查点生成技术,对数据一致性保证提出了新的挑战。多个数据版本的隔离和维护,在程序执行和生成检查点过程重合的情况下变得尤为复杂;与此同时,硬件实现需要简单的逻辑设计。为此提出并利用状态机表达了故障时数据一致性协议;对代码级实现进行了符号抽象,利用不变式和数学归纳法对故障时数据一致性协议的正确性进行了形式化证明。
\end{itemize}

\end{cabstract}

\ckeywords{内存, 非易失性, 持久化, 事务性内存, 检查点}

\begin{eabstract}

Non-volatile memories (NVMs), such as flash, phase-change memory (PCM) and ReRAM, feature both the persistence property of external storage and the high performance of internal memory. They promise an emerging tier in the memory and storage stack.

Non-volatile memory systems ensure durability of memory data on system failures such as power outages and system crashes. This distinctive feature of NVM systems can reduce the overhead of traditional data persistence mechanisms, but introduces the critical challenge of supporting crash consistency of memory data. In order to guarantee such consistency, programs are typically limited in accessing memory data by a certain form of access interfaces. The interface choice and its corresponding consistency mechanism determine the tradeoff between programming ease and system performance.

This dissertation researches efficient consistency mechanisms and/or system performance optimizations for NVM systems, under three main forms of interfaces, file system, transactional memory, and the software-transparent. The main contributions of this dissertation include the following.

\begin{itemize}
\item For file systems, we introduce atomic transactions to the page cache of operating systems to ensure that memory data is flushed to persistent storage in a consistent manner. This technique is applied to smartphones whose DRAM can be assumed as a NVM system, in order to optimize energy efficiency and app responsiveness. We design app-adaptive policies and algorithms to quantatively trade off between data staleness and energy efficiency/app responsiveness.

\item For transactional memories, we propose a new buffering and group commit design, the small buffer array, according to the characteristics of (potentially NVRAM-enhanced) NVM Express-attached flash cards. It largely reduces the waiting latency in group commit, while saturating the bandwidth of the flash cards. Moreover, we employ snapshot isolation to hide write latency from the critical path of read-only transactions, which brings significant performance improvement to real-time analytical workloads.

\item With software-transparent interfaces, programs can safely access memory data using regular load/store instructions. Programmers do not need bother partitioning transient and persistent data or writing transactional code, and enjoy better portability than using particular transaction libraries. To enable this approach, we propose a highly efficient consistent cooperative checkpointing mechanism which synchronously checkpoints memory data at different granularities.

\end{itemize}

\end{eabstract}

\ekeywords{memory, persistence, non-volatile, transactional memory,
checkpointing}

